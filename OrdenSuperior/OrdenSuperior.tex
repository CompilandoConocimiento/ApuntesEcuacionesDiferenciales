% ****************************************************************************************
% ************************      ECUACIONES ORDEN SUPERIOR     ****************************
% ****************************************************************************************


% =======================================================
% =======   ALL COMMANDS AND RULES FOR DOC   ============
% =======================================================
\documentclass[12pt]{report}                               %Type of docuemtn and size of font
\usepackage[margin=1.2in]{geometry}                         %Margins

\usepackage[spanish]{babel}                                 %Please use spanish
\usepackage[utf8]{inputenc}                                 %Please use spanish 
\usepackage[T1]{fontenc}                                    %Please use spanish

\usepackage{amsthm, amssymb, amsfonts}                      %Make math beautiful
\usepackage[fleqn]{amsmath}                                 %Please make equations left
\decimalpoint                                               %Make math beautiful
\setlength{\parindent}{0pt}                                 %Eliminate ugly indentation

\usepackage{graphicx}                                       %Allow to create graphics
\usepackage{wrapfig}                                        %Allow to create images
\graphicspath{ {Graphics/} }                                %Where are the images :D
\usepackage{listings}                                       %We will be using code here
\usepackage[inline]{enumitem}                               %We will need to enumarate

\usepackage{fancyhdr}                                       %Lets make awesome headers/footers
\usepackage{longtable}                                      %Lets make tables awesome

\renewcommand{\footrulewidth}{0.5pt}                        %We will need this!
\setlength{\headheight}{16pt}                               %We will need this!
\setlength{\parskip}{0.5em}                                 %We will need this!
\pagestyle{fancy}                                           %Lets make awesome headers/footers
\lhead{\footnotesize{\leftmark}}                            %Headers!
\rhead{\footnotesize{\rightmark}}                           %Headers!
\lfoot{Compilando Conocimiento}                             %Footers!
\rfoot{Oscar Rosas}                                         %Footers!



\author{Oscar Andrés Rosas}                                 %Who I am




% =====================================================
% ============        COVER PAGE       ================
% =====================================================
\begin{document}
\begin{titlepage}

    \center
    % ============ UNIVERSITY NAME AND DATA =========
    \textbf{\textsc{\Large Proyecto Compilando Conocimiento}}\\[1.0cm] 
    \textsc{\Large Ecuaciones Diferenciales}\\[1.0cm] 

    % ============ NAME OF THE DOCUMENT  ============
    \rule{\linewidth}{0.5mm} \\[1.0cm]
        { \huge \bfseries Ecuaciones de Orden Superior}\\[1.0cm] 
    \rule{\linewidth}{0.5mm} \\[2.0cm]
    
    % ====== SEMI TITLE ==========
    {\LARGE Variación de Parámetros}\\[1cm] 
    {\LARGE Método del Anulador}\\[1cm] 
    {\LARGE Variación de Parámetros}\\[3cm] 
    
    % ============  MY INFORMATION  =================
    \begin{center} \large
    \textbf{\textsc{Autor:}}\\
    Rosas Hernandez Oscar Andres
    \end{center}

    \vfill

\end{titlepage}


% =====================================================
% ========                INDICE              =========
% =====================================================
\tableofcontents{}
\clearpage

% ======================================================================================
% =============================   COEFICIENTES INDETERMINADOS    =======================
% ======================================================================================
\chapter{Coeficientes Indeterminados}
    \clearpage

    % =====================================================
    % ============           DEFINICION            ========
    % =====================================================
    \section{Definición}

        Supon que tienes la siguiente ecuación del estilo:

        \begin{equation}
            \frac{d^ny}{d^nx} + a_{n-1}\frac{d^{n-1}y}{d^{n-1}x} + \cdots + a_2y'' + a_1y' + a_0y = r(x)
        \end{equation}

        Entonces sabemos que nuestra solución esta formada por:
        \begin{equation}
            y = y_h + y_p
        \end{equation}

        Donde sabemos que:
        \begin{itemize}
            \item $y_h$ Es la solución a la ecuación homogenea auxiliar (que ya deberias saber)
            \item $y_p$ Es la solución particular a $r(x)$
        \end{itemize}


    % =====================================================
    % ============        TABLA DE R(X)           ========
    % =====================================================
    \section{Forma de $y_p$ según $r(x)$}

        Donde recuerda nuestra notación:
        \begin{itemize}
            \item Coeficientes Determinados: $a_x, b_x, m, n, q, s$
            \item Son Coeficientes \textbf{Indeterminados}: $A_x , B_x$ 
            \item Es la multiplicidad de cierta raíz: $k$
        \end{itemize}

        \begin{longtable}{p{70mm} || p{90mm}}
            \renewcommand{\arraystretch}{1.5}
            \Large Forma de $r(x)$ & \Large Forma de $y_p$      \\ [0.5ex] 
            \hline\hline                                        \\
            \endfirsthead
            \Large Forma de $r(x)$ & \Large Forma de $y_p$      \\ [0.5ex] 
            \hline\hline                                        \\
            \endhead
        
            % === FORMA DE R(X) ===
            $a_n x^n + a_{n-1}x^{n-1} + \cdots + a_1 x + a_0$                                       &

            % === FORMA DE Y_P ===
            $x^k \left[ A_n x^n + A_{n-1}x^{n-1} + \cdots + A_1 x + A_0 \right]$                    \newline
            \tiny                                                                                   \newline
            \footnotesize Donde k = El num. de veces que 0 es raíz del Polinomio Caracteristico     \\ [6.0ex]

            % --------------------------------------
         
            % === FORMA DE R(X) ===
            $ae^{sx}$                                                                               &

            % === FORMA DE Y_P ===
            $x^k \left[ Ae^{sx} \right]$                                                            \newline
            \tiny                                                                                   \newline
            \footnotesize Donde k = El num. de veces que s es raíz del Polinomio Caracteristico     \\ [6.0ex]

            % --------------------------------------

            % === FORMA DE R(X) ===
            $\left[ a_n x^n + \cdots + a_0 \right]e^{sx}$                                                                                        &

            % === FORMA DE Y_P ===
            $x^k \left[ A_n x^n + \cdots + A_0 \right]e^{sx}$                                       \newline
            \tiny                                                                                   \newline
            \footnotesize Donde k = El num. de veces que s es raíz del Polinomio Caracteristico     \\ [6.0ex]

            % --------------------------------------

            % === FORMA DE R(X) ===
            $a \cos(qx) + b \sin(qx)$                                                               &

            % === FORMA DE Y_P ===
            $x^k \left[ A \cos(qx) + B \sin(qx) \right]$                                            \newline
            \tiny                                                                                   \newline
            \footnotesize Donde k = El num. de veces que $qi$ es raíz del Polinomio Caracteristico  \\ [6.0ex]
 
            % --------------------------------------

            % === FORMA DE R(X) ===
            $\left[a_nx^n+\cdots+a_0\right] \sen(qx) +                                              \newline
            \left[b_nx^n+\cdots+b_0\right]  \cos(qx) $                                              &

            % === FORMA DE Y_P ===
            $x^k \left[A_Nx^N+\cdots+A_0\right]\sen(qx) +                                           \newline
             x^k \left[B_Nx^N+\cdots+B_0\right]\cos(qx) $                                           \newline
            \tiny                                                                                   \newline
            \footnotesize Donde k = El num. de veces que $qi$ es raíz del
            Polinomio Caracteristico y donde N es el máximo entre m y n                             \\ [6.0ex]
         
            % --------------------------------------

            % === FORMA DE R(X) ===
            $a e^{sx} \sen(qx) + b e^{sx} \sen(qx)$                                                 &

            % === FORMA DE Y_P ===
            $x^k \left[A e^{sx} \sen(qx) + B e^{sx} \sen(qx) \right]$                               \newline
            \tiny                                                                                   \newline
            \footnotesize Donde k = El num. de veces que $s+qi$ es raíz del
            Polinomio Caracteristico.                                                               \\ [6.0ex]

            % --------------------------------------
         
            % === FORMA DE R(X) ===
            $\left[a_nx^n+\cdots+a_0\right]e^{sx}\sen(qx) +                                         \newline
            \left[b_nx^n+\cdots+b_0\right]e^{sx}\cos(qx)$                                           &

            % === FORMA DE Y_P ===
            $x^k \left[A_Nx^N+\cdots+A_0\right]e^{sx}\sen(qx) +                                     \newline
             x^k \left[B_Nx^N+\cdots+B_0\right]e^{sx}\cos(qx) $                                     \newline
            \tiny                                                                                   \newline
            \footnotesize Donde k = El num. de veces que $s+qi$ es raíz del
            Polinomio Caracteristico y donde N es el máximo entre m y n                             \\ [6.0ex]
         
        \end{longtable}


        % ================
        % ===  EJEMPLO ===
        % ================
        \clearpage
        \subsubsection{Ejemplo}
            Podemos hacer un ejemplo muy rápido encontrando $y$ en la ecuación:
            \begin{equation*}
                y'' + 9y -9 = 0
            \end{equation*}
            
            Recuerda primero que lo podemos reordenar como:
            \begin{equation*}
                y'' +9y = 9
            \end{equation*}

            Podemos encontrar las raíces del polinomio carácteristicos como:
            \begin{equation*}
                r^2 + 9 = 0 \to r_{1,2} = \pm 3i
            \end{equation*}

            Por lo tanto nuestra parte homogenea es:
            \begin{equation*}
                y_h = c_1 e^{3ix} + c_2 e^{-3ix} 
            \end{equation*}

            Por la tanto tambien ya podemos conocer la parte particular como:
            \begin{equation*}
                y_p = A \to y_p = 1
            \end{equation*}

            Finalmenete tendremos que:
            \begin{equation*}
                y = c_1 e^{3ix} + c_2 e^{-3ix} + 1
            \end{equation*}

        % ================
        % ===  EJEMPLO ===
        % ================
        \clearpage
        \subsubsection{Ejemplo}
            Podemos hacer un ejemplo muy rápido encontrando $y$ en la ecuación:
            \begin{equation*}
                y''' + y'' = 1
            \end{equation*}

            Podemos encontrar las raíces del polinomio carácteristicos como:
            \begin{equation*}
                r^3 + r^2 = 0 \to r^2(r+1) \to r_{1,2} = 0  \quad r_{3} = -1 
            \end{equation*}

            Por lo tanto nuestra parte homogenea es:
            \begin{equation*}
                y_h = c_1 + c_2x + c_3e^{-x}
            \end{equation*}

            Por la tanto tambien ya podemos conocer la parte particular como:
            \begin{equation*}
                y_p   = x^2A \\
                y_p'  = 2xA  \\
                y_p'' = 2A   \\
                y_p'''= 0   
            \end{equation*}

            Ahora vamos a sustituir en la ecuación nuestros descubrimientos para conocer a $A$:
            \begin{equation*}
                0 + 2A = 1 \quad \therefore \quad A = \frac{1}{2}
            \end{equation*}

            Finalmenete tendremos que:
            \begin{equation*}
                c_1 + c_2x + c_3e^{-x} + \frac{x^2}{2}
            \end{equation*}

% ======================================================================================
% =============================      REDUCCION DE ORDEN       ==========================
% ======================================================================================
\chapter{Reducción de Orden}
    \clearpage

    % =====================================================
    % ============           DEFINICION            ========
    % =====================================================
    \section{Definición}

        Veamos la siguiente solución general de la siguiente ecuación:

        \begin{equation}
            \frac{d^2 x}{dt^2} + 2 \lambda \frac{dx}{dt} + w^2 x = F_0 sen(\gamma t)
        \end{equation}

        La podemos escribir suponiendo que $\lambda ^2 - w^2 < 0$
        \begin{equation*}
            x(t) = A e^{-\lambda t} sen\left[ \sqrt{w^2-\lambda^2} t + \phi \right] + \frac{F_0}{\sqrt{(w^2-\gamma^2)^2 + 4\lambda^2 \gamma^2} } sen(\gamma t + \theta)
        \end{equation*}

        Donde podemos saber que:
        \begin{equation*}
            A = \sqrt{C_1^2 + C_2^2}
        \end{equation*}

        \begin{equation*}
            sen(\phi) = \frac{C_1}{A}
        \end{equation*}

        \begin{equation*}
            sen(\phi) = \frac{C_1}{A}
        \end{equation*}

        \begin{equation*}
            sen(\theta) = \frac{-2\lambda \gamma}{\sqrt{(w^2-\gamma^2)^2 +4\lambda^2\gamma^2 }}
        \end{equation*}

        \begin{equation*}
            cos(\theta) = \frac{w^2 - \gamma^2}{\sqrt{(w^2-\gamma^2)^2 +4\lambda^2\gamma^2 }}
        \end{equation*}


% ======================================================================================
% =============================   ECUACIONES DE CAUCHY-EULER    ========================
% ======================================================================================
\chapter{Ecuaciones de Cauchy - Euler}
    \clearpage

    % =====================================================
    % ============           DEFINICION            ========
    % =====================================================
    \section{Definición}

        Veamos la siguiente ecuación:

        \begin{equation}
            a_n x^n \frac{d^n y}{dx^n} + a_{n-1} x^{n-1} \frac{d^{n-1} y}{dx^{n-1}} + \cdots + a_1 x \frac{dy}{dx} + a_0 y = g(x)
        \end{equation}

        Donde los coeficientes $a_n, a_{n-1}, \cdots , a_1, a_0$ son constante, entonces la ecuación se conoce como Ecuación de Cauchy - Euler.

        \subsection{Solución}
        Podemos solucionarla propiendo una función del estilo $y = x^m$ con m como incognita.

        Usando mas formulazo podemos dividir los casos de las soluciones en 3 casos:

        \begin{itemize}
            \item Raices Reales con $r_1 \neq r_2$ donde la solución es $ y = C_1 x ^{r_1} + C_2 x^{r_2}$ 
            \item Raices Iguales con $r_1 = r_2$ donde la solución es $ y = C_1 x ^{r_1} + C_2 ln(x) \cdot x^{r_1}$ 
            \item Raices Complejas del estilo $r_x  = a \pm ib$ donde la solución es \\
            $ y = C_1 x^a cos(b \cdot ln(x)) + C_2 x^a sen(b \cdot ln(x))$ 
        \end{itemize}


        Otro formulazo útil es que si tenemos de manera general una raiz de multiplicidad m tenemos que la solución sera:
        \begin{equation}
            x^r (ln(x))^m
        \end{equation}


    % ==========================
    % ====    EJEMPLO  =========
    % ==========================
        \subsection{Ejemplo 1}
        \begin{equation*}
            x^2 \frac{d^2 y}{dx^2} - 2x \frac{dy}{dx} - 4y = 0
        \end{equation*}

        Podemos proponer nuestra función como:
        $y = x^m$\\
        $y' = mx^{m-1}$\\
        $y'' = (m)(m-1)x^{m-2}$\\

        Entonces tendremos que:
        \begin{equation*}
            x^2 (m)(m-1)x^{m-2} - 2x mx^{m-1} - 4x^m = 0
        \end{equation*}

        \begin{equation*}
            x^m [(m)(m-1)-  2m -4] = 0
        \end{equation*}
        Podemos saber entonces que las raices son : $ m = 4$ y $m = -1$

        Así la solución sera:

        $y = C_1X^4 + C_2 X^{-1}$

    % ==========================
    % ====    EJEMPLO  =========
    % ==========================
        \subsection{Ejemplo 2}
        \begin{equation*}
            x^2 \frac{d^2 y}{dx^2} + x \frac{dy}{dx} = 0
        \end{equation*}

        Podemos proponer nuestra función como:
        $y = x^m$\\
        $y' = mx^{m-1}$\\
        $y'' = (m)(m-1)x^{m-2}$\\

        Entonces tendremos que:
        \begin{equation*}
            x^2 (m)(m-1)x^{m-2} + x mx^{m-1} = 0
        \end{equation*}

        \begin{equation*}
            x^m [m^2 - m + m] = 0 = x^m [m^2] = 0
        \end{equation*}
        Podemos saber entonces que las raices son : $ m_1 = 0$ y $m_2 = 0$

        Así la solución sera:
        \begin{equation*}
            y = C_1 x ^{r_1} + C_2 ln(x) \cdot x^{r_1}
        \end{equation*}

        Que se puede simplificar como:
        \begin{equation*}
            y = C_1 + C_2 ln(x)
        \end{equation*}

    % ==========================
    % ====    EJEMPLO  =========
    % ==========================
        \subsection{Ejemplo 3}
        \begin{equation*}
            x^2 y'' + 7 x y' + 13y = 0
        \end{equation*}

        Podemos proponer nuestra función como:
        $y = x^m$\\
        $y' = mx^{m-1}$\\
        $y'' = (m)(m-1)x^{m-2}$\\

        Entonces tendremos que:
        \begin{equation*}
            x^2 \cdot (m)(m-1)x^{m-2} + 7x \cdot mx^{m-1} + 13 x^m= 0
        \end{equation*}

        Que podemos simplificar como:
        \begin{equation*}
            x^m [m^2 + 6m + 13] = 0
        \end{equation*}

        \begin{equation*}
            m_{1,2} = \frac{-6 \pm \sqrt{36-52}}{2} = \frac{-6 \pm 4i}{2} = -3 \pm 2i
        \end{equation*}
        Podemos saber entonces que las raices son : $ -3 \pm 2i$

        Así la solución sera:
        \begin{equation*}
            y = C_1 x^a cos(b \cdot ln(x)) + C_2 x^a sen(b \cdot ln(x))
        \end{equation*}

        Que se puede simplificar como:
        \begin{equation*}
            y = C_1 x^{-3} cos(2 \cdot ln(x)) + C_2 x^{-3} sen(2 \cdot ln(x))
        \end{equation*}

    % ==========================
    % ====    EJEMPLO  =========
    % ==========================
        \subsection{Ejemplo 4}
        \begin{equation*}
            x^3 y''' -x^2y'' + xy' = 0
        \end{equation*}

        Podemos proponer nuestra función como:
        $y = x^m$\\
        $y' = mx^{m-1}$\\
        $y'' = (m)(m-1)x^{m-2}$\\
        $y''' = (m)(m-1)(m-2)x^{m-3}$\\

        Entonces tendremos que:
        \begin{equation*}
            x^3 \cdot (m)(m-1)(m-2)x^{m-3} -x^2 \cdot (m)(m-1)x^{m-2} + x \cdot mx^{m-1} = 0
        \end{equation*}

        Que podemos simplificar como:
        \begin{equation*}
            x^m [(m)(m-1)(m-2) - (m)(m-1) + m ] = x^m[(m^2-m)(m-2) - (m^2-m) + m] 
                                                = x^m[ m^3 - 4m^2 + 4m] 
        \end{equation*}
        Podemos saber entonces que las raices son : $m_1 = 0$ y para las otras dos $m_{2,3} = 2$

        Así la solución sera:
        \begin{equation*}
            y = C_1 x ^{r_1} + C_2 x^2 + C_3 x^2 ln(x)
        \end{equation*}


% ======================================================================================
% =============================    ????                        ==========================
% ======================================================================================
\chapter{?????}
    \clearpage

    Podemos escribir nuestra ecuación de esta manera:
    \begin{equation}
        y'' + P(x) y' + Q(x) y = 0
    \end{equation}

    Donde la solución esta dada por:
    \begin{equation}
        y = y_1(x) \int \frac{ e^{-\int P(x) dx} }{y_1^2(x)} dx
    \end{equation}

    % ==========================
    % ====    EJEMPLO  =========
    % ==========================
    \subsubsection{Ejemplo 1}
    La función $y_1 = x^2$ es una solución de $x^2 y'' + 3xy' +4y = 0$
    Encuentre la solución general de ecuación general. \\

    Podemos reescribirla como:
    $y'' + \frac{3}{x}y' + \frac{4}{x^2}y = 0 $

    Podemos ver que $P(x) = \frac{3}{x}$

    Por lo tanto tenemos que:
    \begin{equation*}
        \int P(x) dx = -\ln (x^3)
    \end{equation*}

    Y ya sustituyendo en nuestra formula maestra:
    \begin{equation*}
        y = x^2 \int \frac{ e^{-ln(x^3)} }{(x^2)^2(x)} dx = x^2 \int \frac{x^3}{x^4} = x^2 ln(x)
    \end{equation*}

    Por lo tanto nuestro resultado será:
    \begin{equation*}
        \phi(x) = C_1 x^2 + C_2 x^2 ln(x)
    \end{equation*}



% =====================================================
% ============        BIBLIOGRAPHY   ==================
% =====================================================
\clearpage
\bibliographystyle{plain}
    \begin{thebibliography}{9}

    % ============ REFERENCE #1 ========
    \bibitem{Sitio1} 
        ProbRob
        \\\texttt{Youtube.com}


     

\end{thebibliography}



\end{document}