% ****************************************************************************************
% ************************     	ECUACIONES ORDEN SUPERIOR	  ****************************
% ****************************************************************************************


% =======================================================
% =======	ALL COMMANDS AND RULES FOR DOC 	 ============
% =======================================================
\documentclass[12pt]{article}							    %Type of docuemtn and size of font
\usepackage[margin=1.2in]{geometry}							%Margins

\usepackage[spanish]{babel}									%Please use spanish
\usepackage[utf8]{inputenc}									%Please use spanish	
\usepackage[T1]{fontenc}									%Please use spanish

\usepackage{amsthm, amssymb, amsfonts}				        %Make math beautiful
\usepackage[fleqn]{amsmath}                                 %Please make equations left
\decimalpoint												%Make math beautiful
\setlength{\parindent}{0pt}									%Eliminate ugly indentation

\usepackage{graphicx}										%Allow to create graphics
\usepackage{wrapfig}                                    	%Allow to create images
\graphicspath{ {Graphics/} }                                %Where are the images :D
\usepackage{listings}										%We will be using code here
\usepackage[inline]{enumitem}								%We will need to enumarate

\usepackage{fancyhdr}										%Lets make awesome headers/footers
\renewcommand{\footrulewidth}{0.5pt}						%We will need this!
\setlength{\headheight}{16pt} 								%We will need this!
\setlength{\parskip}{0.5em}									%We will need this!
\pagestyle{fancy}											%Lets make awesome headers/footers
\lhead{\footnotesize{\leftmark}}							%Headers!
\rhead{\footnotesize{\rightmark}}							%Headers!
\lfoot{Compilando Conocimiento}								%Footers!
\rfoot{Oscar Rosas}						                    %Footers!

\author{Oscar Andrés Rosas}						            %Who I am




% =====================================================
% ============     	  COVER PAGE	   ================
% =====================================================
\begin{document}
\begin{titlepage}

	\center
	% ============ UNIVERSITY NAME AND DATA =========
	\textbf{\textsc{\Large Proyecto Compilando Conocimiento}}\\[1.0cm] 
	\textsc{\Large Calculo}\\[1.0cm] 

	% ============ NAME OF THE DOCUMENT  ============
	\rule{\linewidth}{0.5mm} \\[1.0cm]
		{ \huge \bfseries Ecuaciones de Orden Superior}\\[1.0cm] 
	\rule{\linewidth}{0.5mm} \\[2.0cm]
	
	% ====== SEMI TITLE ==========
	{\LARGE Variación de Parámetros y Método del Anulador}\\[7cm] 
	
	% ============  MY INFORMATION  =================
	\begin{center} \large
	\textbf{\textsc{Autor:}}\\
	Rosas Hernandez Oscar Andres
	\end{center}

	\vfill

\end{titlepage}




% ======================================================================================
% ==================================     DOCUMENT  =====================================
% ======================================================================================



% =====================================================
% ========              ?????           =========
% =====================================================
\section{????}


% ==========================
% ==== EXPLICACION =========
% ==========================
\subsection{¿Qué es?}

Podemos escribir nuestra ecuación de esta manera:
\begin{equation}
    y'' + P(x) y' + Q(x) y = 0
\end{equation}


Donde la solución esta dada por:
\begin{equation}
    y = y_1(x) \int \frac{ e^{-\int P(x) dx} }{y_1^2(x)} dx
\end{equation}

% ==========================
% ====    EJEMPLO  =========
% ==========================
\subsection{Ejemplo 1}
La función $y_1 = x^2$ es una solución de $x^2 y'' + 3xy' +4y = 0$
Encuentre la solución general de ecuación general. \\

Podemos reescribirla como:
$y'' + \frac{3}{x}y' + \frac{4}{x^2}y = 0 $

Podemos ver que $P(x) = \frac{3}{x}$

Por lo tanto tenemos que:
\begin{equation*}
    \int P(x) dx = -\ln (x^3)
\end{equation*}

Y ya sustituyendo en nuestra formula maestra:
\begin{equation*}
    y = x^2 \int \frac{ e^{-ln(x^3)} }{(x^2)^2(x)} dx = x^2 \int \frac{x^3}{x^4} = x^2 ln(x)
\end{equation*}

Por lo tanto nuestro resultado será:
\begin{equation*}
    \phi(x) = C_1 x^2 + C_2 x^2 ln(x)
\end{equation*}




% =====================================================
% ========        REDUCCION DE ORDEN          =========
% =====================================================
\section{Reducción de Orden}


% ==========================
% ==== EXPLICACION =========
% ==========================
\subsection{¿Qué es?}
Veamos la siguiente solución general de la siguiente ecuación:

\begin{equation}
	\frac{d^2 x}{dt^2} + 2 \lambda \frac{dx}{dt} + w^2 x = F_0 sen(\gamma t)
\end{equation}

La podemos escribir suponiendo que $\lambda ^2 - w^2 < 0$
\begin{equation*}
	x(t) = A e^{-\lambda t} sen\left[ \sqrt{w^2-\lambda^2} t + \phi \right] + \frac{F_0}{\sqrt{(w^2-\gamma^2)^2 + 4\lambda^2 \gamma^2} } sen(\gamma t + \theta)
\end{equation*}

Donde podemos saber que:
\begin{equation*}
	A = \sqrt{C_1^2 + C_2^2}
\end{equation*}

\begin{equation*}
	sen(\phi) = \frac{C_1}{A}
\end{equation*}

\begin{equation*}
	sen(\phi) = \frac{C_1}{A}
\end{equation*}

\begin{equation*}
	sen(\theta) = \frac{-2\lambda \gamma}{\sqrt{(w^2-\gamma^2)^2 +4\lambda^2\gamma^2 }}
\end{equation*}

\begin{equation*}
	cos(\theta) = \frac{w^2 - \gamma^2}{\sqrt{(w^2-\gamma^2)^2 +4\lambda^2\gamma^2 }}
\end{equation*}




% =====================================================
% ========     ECUACIONES DE CAUCHY-EULER     =========
% =====================================================
	\clearpage
	\section{Ecuaciones de Cauchy - Euler}

	% ==========================
	% ==== EXPLICACION =========
	% ==========================
		\subsection{¿Qué es?}
		Veamos la siguiente ecuación:

		\begin{equation}
			a_n x^n \frac{d^n y}{dx^n} + a_{n-1} x^{n-1} \frac{d^{n-1} y}{dx^{n-1}} + \cdots + a_1 x \frac{dy}{dx} + a_0 y = g(x)
		\end{equation}

		Donde los coeficientes $a_n, a_{n-1}, \cdots , a_1, a_0$ son constante, entonces la ecuación se conoce como Ecuación de Cauchy - Euler.

		\subsection{Solución}
		Podemos solucionarla propiendo una función del estilo $y = x^m$ con m como incognita.

		Usando mas formulazo podemos dividir los casos de las soluciones en 3 casos:

		\begin{itemize}
			\item Raices Reales con $r_1 \neq r_2$ donde la solución es $ y = C_1 x ^{r_1} + C_2 x^{r_2}$ 
			\item Raices Iguales con $r_1 = r_2$ donde la solución es $ y = C_1 x ^{r_1} + C_2 ln(x) \cdot x^{r_1}$ 
			\item Raices Complejas del estilo $r_x  = a \pm ib$ donde la solución es \\
			$ y = C_1 x^a cos(b \cdot ln(x)) + C_2 x^a sen(b \cdot ln(x))$ 
		\end{itemize}


		Otro formulazo útil es que si tenemos de manera general una raiz de multiplicidad m tenemos que la solución sera:
		\begin{equation}
			x^r (ln(x))^m
		\end{equation}


	% ==========================
	% ====    EJEMPLO  =========
	% ==========================
		\subsection{Ejemplo 1}
		\begin{equation*}
			x^2 \frac{d^2 y}{dx^2} - 2x \frac{dy}{dx} - 4y = 0
		\end{equation*}

		Podemos proponer nuestra función como:
		$y = x^m$\\
		$y' = mx^{m-1}$\\
		$y'' = (m)(m-1)x^{m-2}$\\

		Entonces tendremos que:
		\begin{equation*}
			x^2 (m)(m-1)x^{m-2} - 2x mx^{m-1} - 4x^m = 0
		\end{equation*}

		\begin{equation*}
			x^m [(m)(m-1)-  2m -4] = 0
		\end{equation*}
		Podemos saber entonces que las raices son : $ m = 4$ y $m = -1$

		Así la solución sera:

		$y = C_1X^4 + C_2 X^{-1}$

	% ==========================
	% ====    EJEMPLO  =========
	% ==========================
		\subsection{Ejemplo 2}
		\begin{equation*}
			x^2 \frac{d^2 y}{dx^2} + x \frac{dy}{dx} = 0
		\end{equation*}

		Podemos proponer nuestra función como:
		$y = x^m$\\
		$y' = mx^{m-1}$\\
		$y'' = (m)(m-1)x^{m-2}$\\

		Entonces tendremos que:
		\begin{equation*}
			x^2 (m)(m-1)x^{m-2} + x mx^{m-1} = 0
		\end{equation*}

		\begin{equation*}
			x^m [m^2 - m + m] = 0 = x^m [m^2] = 0
		\end{equation*}
		Podemos saber entonces que las raices son : $ m_1 = 0$ y $m_2 = 0$

		Así la solución sera:
		\begin{equation*}
			y = C_1 x ^{r_1} + C_2 ln(x) \cdot x^{r_1}
		\end{equation*}

		Que se puede simplificar como:
		\begin{equation*}
			y = C_1 + C_2 ln(x)
		\end{equation*}



	% ==========================
	% ====    EJEMPLO  =========
	% ==========================
		\subsection{Ejemplo 3}
		\begin{equation*}
			x^2 y'' + 7 x y' + 13y = 0
		\end{equation*}

		Podemos proponer nuestra función como:
		$y = x^m$\\
		$y' = mx^{m-1}$\\
		$y'' = (m)(m-1)x^{m-2}$\\

		Entonces tendremos que:
		\begin{equation*}
			x^2 \cdot (m)(m-1)x^{m-2} + 7x \cdot mx^{m-1} + 13 x^m= 0
		\end{equation*}

		Que podemos simplificar como:
		\begin{equation*}
			x^m [m^2 + 6m + 13] = 0
		\end{equation*}

		\begin{equation*}
			m_{1,2} = \frac{-6 \pm \sqrt{36-52}}{2} = \frac{-6 \pm 4i}{2} = -3 \pm 2i
		\end{equation*}
		Podemos saber entonces que las raices son : $ -3 \pm 2i$

		Así la solución sera:
		\begin{equation*}
			y = C_1 x^a cos(b \cdot ln(x)) + C_2 x^a sen(b \cdot ln(x))
		\end{equation*}

		Que se puede simplificar como:
		\begin{equation*}
			y = C_1 x^{-3} cos(2 \cdot ln(x)) + C_2 x^{-3} sen(2 \cdot ln(x))
		\end{equation*}

	% ==========================
	% ====    EJEMPLO  =========
	% ==========================
		\subsection{Ejemplo 4}
		\begin{equation*}
			x^3 y''' -x^2y'' + xy' = 0
		\end{equation*}

		Podemos proponer nuestra función como:
		$y = x^m$\\
		$y' = mx^{m-1}$\\
		$y'' = (m)(m-1)x^{m-2}$\\
		$y''' = (m)(m-1)(m-2)x^{m-3}$\\

		Entonces tendremos que:
		\begin{equation*}
			x^3 \cdot (m)(m-1)(m-2)x^{m-3} -x^2 \cdot (m)(m-1)x^{m-2} + x \cdot mx^{m-1} = 0
		\end{equation*}

		Que podemos simplificar como:
		\begin{equation*}
			x^m [(m)(m-1)(m-2) - (m)(m-1) + m ] = x^m[(m^2-m)(m-2) - (m^2-m) + m] 
												= x^m[ m^3 - 4m^2 + 4m] 
		\end{equation*}
		Podemos saber entonces que las raices son : $m_1 = 0$ y para las otras dos $m_{2,3} = 2$

		Así la solución sera:
		\begin{equation*}
			y = C_1 x ^{r_1} + C_2 x^2 + C_3 x^2 ln(x)
		\end{equation*}



% =====================================================
% ============        BIBLIOGRAPHY   ==================
% =====================================================
\clearpage
\bibliographystyle{plain}
	\begin{thebibliography}{9}

	% ============ REFERENCE #1 ========
	\bibitem{Sitio1} 
		ProbRob
		\\\texttt{Youtube.com}


	 

\end{thebibliography}



\end{document}