% ****************************************************************************************
% *************    	APLICACIONES DE ECUACIONES DIFERENCIALES	  ************************
% ****************************************************************************************


% =======================================================
% =======	ALL COMMANDS AND RULES FOR DOC 	 ============
% =======================================================
\documentclass[12pt]{article}							    %Type of docuemtn and size of font
\usepackage[margin=1.2in]{geometry}							%Margins

\usepackage[spanish]{babel}									%Please use spanish
\usepackage[utf8]{inputenc}									%Please use spanish	
\usepackage[T1]{fontenc}									%Please use spanish

\usepackage{amsthm, amssymb, amsfonts}				        %Make math beautiful
\usepackage[fleqn]{amsmath}                                 %Please make equations left
\decimalpoint												%Make math beautiful
\setlength{\parindent}{0pt}									%Eliminate ugly indentation

\usepackage{graphicx}										%Allow to create graphics
\usepackage{wrapfig}                                    	%Allow to create images
\graphicspath{ {Graphics/} }                                %Where are the images :D
\usepackage{listings}										%We will be using code here
\usepackage[inline]{enumitem}								%We will need to enumarate

\usepackage{fancyhdr}										%Lets make awesome headers/footers
\renewcommand{\footrulewidth}{0.5pt}						%We will need this!
\setlength{\headheight}{16pt} 								%We will need this!
\setlength{\parskip}{0.5em}									%We will need this!
\pagestyle{fancy}											%Lets make awesome headers/footers
\lhead{\footnotesize{\leftmark}}							%Headers!
\rhead{\footnotesize{\rightmark}}							%Headers!
\lfoot{Compilando Conocimiento}								%Footers!
\rfoot{Oscar Rosas}						                    %Footers!

\author{Oscar Andrés Rosas}						            %Who I am




% =====================================================
% ============     	  COVER PAGE	   ================
% =====================================================
\begin{document}
\begin{titlepage}

	\center
	% ============ UNIVERSITY NAME AND DATA =========
	\textbf{\textsc{\Large Proyecto Compilando Conocimiento}}\\[1.0cm] 
	\textsc{\Large Calculo}\\[1.0cm] 

	% ============ NAME OF THE DOCUMENT  ============
	\rule{\linewidth}{0.5mm} \\[1.0cm]
		{ \huge \bfseries Aplicaciones de Ecuaciones Diferenciales}\\[1.0cm] 
	\rule{\linewidth}{0.5mm} \\[2.0cm]
	
	% ====== SEMI TITLE ==========
	{\LARGE Ecuaciones Diferenciales}\\[7cm] 
	
	% ============  MY INFORMATION  =================
	\begin{center} \large
	\textbf{\textsc{Autor:}}\\
	Rosas Hernandez Oscar Andres
	\end{center}

	\vfill

\end{titlepage}




% ======================================================================================
% ==================================     DOCUMENT  =====================================
% ======================================================================================



% =====================================================
% ========              CIRCUITOS RCL         =========
% =====================================================
\section{Circuitos RCL}

Para trabajar circuitos tenemos que saber algunas cuantas ecuaciones:
\begin{itemize}
	\item Voltaje del Inductor: $V_L = L \frac{di(t)}{dt}$
	\item Voltaje de la Resistencia: $V_R = R i(t)$
	\item Voltaje del Capacitor: $V_C = \frac{C}{q}$
\end{itemize}

Despues podemos saber gracias a la Ley de Kirchhoff que la suma de todas las caidas de voltaje es igual a la fuerza electromotriz:
\begin{equation}
    \sum V_X = E
\end{equation}


% =====================================================
% ============        BIBLIOGRAPHY   ==================
% =====================================================
\clearpage
\bibliographystyle{plain}
	\begin{thebibliography}{9}

	% ============ REFERENCE #1 ========
	\bibitem{Sitio1} 
		ProbRob
		\\\texttt{Youtube.com}


	 

\end{thebibliography}



\end{document}