% ****************************************************************************************
% ************************     	TRANFORMADA DE LAPLACE	  ********************************
% ****************************************************************************************


% =======================================================
% =======	ALL COMMANDS AND RULES FOR DOC 	 ============
% =======================================================
\documentclass[12pt]{article}                               %Type of docuemtn and size of font
\usepackage[margin=1.2in]{geometry}                         %Margins

\usepackage[spanish]{babel}                                 %Please use spanish
\usepackage[utf8]{inputenc}                                 %Please use spanish 
\usepackage[T1]{fontenc}                                    %Please use spanish

\usepackage{amsthm, amssymb, amsfonts}                      %Make math beautiful
\usepackage[fleqn]{amsmath}                                 %Please make equations left
\decimalpoint                                               %Make math beautiful
\setlength{\parindent}{0pt}                                 %Eliminate ugly indentation

\usepackage{graphicx}                                       %Allow to create graphics
\usepackage{wrapfig}                                        %Allow to create images
\graphicspath{ {Graphics/} }                                %Where are the images :D
\usepackage{listings}                                       %We will be using code here
\usepackage[inline]{enumitem}                               %We will need to enumarate

\usepackage{fancyhdr}                                       %Lets make awesome headers/footers
\renewcommand{\footrulewidth}{0.5pt}                        %We will need this!
\setlength{\headheight}{16pt}                               %We will need this!
\setlength{\parskip}{0.5em}                                 %We will need this!
\pagestyle{fancy}                                           %Lets make awesome headers/footers
\lhead{\footnotesize{\leftmark}}                            %Headers!
\rhead{\footnotesize{\rightmark}}                           %Headers!
\lfoot{Compilando Conocimiento}                             %Footers!
\rfoot{Oscar Rosas}                                         %Footers!

\author{Oscar Andrés Rosas}                                 %Who I am



% =====================================================
% ============     	  COVER PAGE	   ================
% =====================================================
\begin{document}
\begin{titlepage}

	\center
	% ============ UNIVERSITY NAME AND DATA =========
	\textbf{\textsc{\Large Proyecto Compilando Conocimiento}}\\[1.0cm] 
	\textsc{\Large Ecuaciones Diferenciales}\\[1.0cm] 

	% ============ NAME OF THE DOCUMENT  ============
	\rule{\linewidth}{0.5mm} \\[1.0cm]
		{ \huge \bfseries La Transformada de Laplace}\\[1.0cm] 
	\rule{\linewidth}{0.5mm} \\[2.0cm]
	
	% ====== SEMI TITLE ==========
	{\LARGE Introducción}\\[7cm] 
	
	% ============  MY INFORMATION  =================
	\begin{center} \large
	\textbf{\textsc{Autor:}}\\
	Rosas Hernandez Oscar Andres
	\end{center}

	\vfill

\end{titlepage}

% =====================================================
% ========                INDICE              =========
% =====================================================
\tableofcontents{}
\clearpage

% ======================================================================================
% =============================         LAPLACE               ==========================
% ======================================================================================
\chapter{La Transformada de Laplace}
    \clearpage

    % =====================================================
    % ============           DEFINICION            ========
    % =====================================================
    \section{Definición}
        Dada una función f(t) definida para toda $t \leq 0$ la tranformada de
        Laplace de f es la función F(s) definida de la Siguiente manera:

        \begin{equation}   
            L\{f(t)\} = F(s) = \int_0^\infty e^{-st} f(t) dt 
        \end{equation}    


        en todos los valoers de S para los cuales la Integral Impropia converge.
        Recuerda que una Integral Impropia:

        \begin{equation*}   
            \int_0^\infty g(t) dt = \lim_{b \to \infty} \int_a^b g(t) dt
        \end{equation*}  



        % ==========================
        % ====    EJEMPLOS    ======
        % ==========================
        \subsection{Ejemplo}
            Calcule la Tranformada de Laplace cuando $f(t) = 1$

            \begin{equation*}   
            \begin{split}
                L\{1\} = F(s) & = \int_0^\infty e^{-st} 1 dt                            \\
                              & = \lim_{b \to \infty} \int_0^b e^{-st} dt               \\
                              & = \lim_{b \to \infty} \frac{e^{-st}}{-s} |_0^b          \\
                              & = \lim_{b \to \infty} \left[ \frac{1}{s} -
                                                        \frac{e^{-sb}}{s} \right]       \\
                              & = \frac{1}{s}
            \end{split}
            \end{equation*}

        % ==========================
        % ====    EJEMPLOS    ======
        % ==========================
        \subsection{Ejemplo}
            Calcule la Tranformada de Laplace cuando $f(t) = e^{at}$

            \begin{equation*}   
            \begin{split}
                L\{1\} = F(s) & = \int_0^\infty e^{-st} \cdot e^{at} dt                     \\
                              & = \lim_{b \to \infty} \int_0^b e^{-st+at} dt                \\
                              & = \lim_{b \to \infty} \int_0^b e^{-(s-a)t} dt               \\
                              & = \lim_{b \to \infty} \frac{e^{-(s-a)t}}{-(s-a)} |_0^b      \\
                              & = \lim_{b \to \infty} \left[ \frac{e^{-(s-a)b}}{-(s-a)} -   
                                                        \frac{e^{-(s-a)0}}{-(s-a)} \right]  \\
                              & = \frac{1}{s-a}
            \end{split}
            \end{equation*}



% =====================================================
% ============        BIBLIOGRAPHY   ==================
% =====================================================
\clearpage
\bibliographystyle{plain}
	\begin{thebibliography}{9}

	% ============ REFERENCE #1 ========
	\bibitem{Sitio1} 
		ProbRob
		\\\texttt{Youtube.com}


	 

\end{thebibliography}



\end{document}
