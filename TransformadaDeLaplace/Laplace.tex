% ****************************************************************************************
% ************************     	TRANFORMADA DE LAPLACE	  ********************************
% ****************************************************************************************


% =======================================================
% =======         HEADER FOR DOCUMENT        ============
% =======================================================
    \documentclass[12pt]{report}                                %Type of docuemtn and size of font
    \usepackage[margin=1.2in]{geometry}                         %Margins
    \usepackage{hyperref}                                       %Create MetaData for a PDF

    \usepackage[spanish]{babel}                                 %Please use spanish
    \usepackage[utf8]{inputenc}                                 %Please use spanish 
    \usepackage[T1]{fontenc}                                    %Please use spanish

    \usepackage{amsthm, amssymb, amsfonts, mathrsfs}            %Make math beautiful
    \usepackage[fleqn]{amsmath}                                 %Please make equations left
    \decimalpoint                                               %Make math beautiful
    \setlength{\parindent}{0pt}                                 %Eliminate ugly indentation

    \usepackage{graphicx}                                       %Allow to create graphics
    \usepackage{wrapfig}                                        %Allow to create images
    \graphicspath{ {Graphics/} }                                %Where are the images :D
    \usepackage{listings}                                       %We will be using code here
    \usepackage[inline]{enumitem}                               %We will need to enumarate

    \usepackage{fancyhdr}                                       %Lets make awesome headers/footers
    \usepackage{tasks}                                          %Horizontal lists
    \usepackage{longtable}                                      %Lets make tables awesome

    \renewcommand{\footrulewidth}{0.5pt}                        %We will need this!
    \setlength{\headheight}{16pt}                               %We will need this!
    \setlength{\parskip}{0.5em}                                 %We will need this!
    \pagestyle{fancy}                                           %Lets make awesome headers/footers
    \lhead{\footnotesize{\leftmark}}                            %Headers!
    \rhead{\footnotesize{\rightmark}}                           %Headers!
    \lfoot{Compilando Conocimiento}                             %Footers!
    \rfoot{Oscar Rosas}                                         %Footers!

    \author{Oscar Andrés Rosas}                                 %Who I am

% ========================================
% ===========   COMMANDS    ==============
% ========================================
    \DeclareMathOperator \Real {\mathbb{R}}                     %The real numbers
    \DeclareMathOperator \Naturals {\mathbb{N}}                 %The real numbers
    \DeclareMathOperator \LinealTransformation {\mathcal{T}}    %A Cool T, that's it!



% =====================================================
% ============     	  COVER PAGE	   ================
% =====================================================
\begin{document}
\begin{titlepage}

	\center
	% ============ UNIVERSITY NAME AND DATA =========
	\textbf{\textsc{\Large Proyecto Compilando Conocimiento}}\\[1.0cm] 
	\textsc{\Large Ecuaciones Diferenciales}\\[1.0cm] 

	% ============ NAME OF THE DOCUMENT  ============
	\rule{\linewidth}{0.5mm} \\[1.0cm]
		{ \huge \bfseries La Transformada de Laplace}\\[1.0cm] 
	\rule{\linewidth}{0.5mm} \\[2.0cm]
	
	% ====== SEMI TITLE ==========
	{\LARGE Introducción}\\[7cm] 
	
	% ============  MY INFORMATION  =================
	\begin{center} \large
	\textbf{\textsc{Autor:}}\\
	Rosas Hernandez Oscar Andres
	\end{center}

	\vfill

\end{titlepage}

% =====================================================
% ========                INDICE              =========
% =====================================================
\tableofcontents{}
\clearpage

% ======================================================================================
% =============================         LAPLACE               ==========================
% ======================================================================================
\chapter{La Transformada de Laplace}
    \clearpage

    % =====================================================
    % ============           DEFINICION            ========
    % =====================================================
    \section{Definición}
        Dada una función f(t) definida para toda $t \geq 0$ la tranformada de
        Laplace de f es la función F(s) definida de la Siguiente manera:

        \begin{equation}   
            \mathscr{L}\{f(t)\} = F(s) = \int_0^\infty e^{-st} f(t) dt 
        \end{equation}    


        en todos los valores de S para los cuales la Integral Impropia converge.
        Recuerda que una Integral Impropia:

        \begin{equation*}   
            \int_0^\infty g(t) dt = \lim_{b \to \infty} \int_a^b g(t) dt
        \end{equation*}  


    % =====================================================
    % =====     TABLA DE TRANSFORMACION            ========
    % =====================================================
    \clearpage
    \section{Tabla de Transformación}

        \begin{longtable}{p{60mm} || p{70mm}}
            \renewcommand{\arraystretch}{1.5}
            \Huge $f(t)$ & \Huge $\mathscr{L}\{f(t)\} = F(s)$                                                       \\ [1.5ex] 
            \hline\hline                                                                                            \\
            \endfirsthead   
            \Huge $f(t)$ & \Huge $\mathscr{L}\{f(t)\} = F(s)$                                                       \\ [1.5ex]
            \hline\hline                                                                   
            \endhead

            \Large $k$                               &\Large      $\dfrac{k}{s}$                                    \\ [3.0ex]
            \Large $t^n$                             &\Large      $\dfrac{n!}{s^{n+1}}$                             \\ [3.0ex]
            \Large $e^{at}$                          &\Large      $\dfrac{1}{s-a}$                                  \\ [3.0ex]
            \Large $sen(kt)$                         &\Large      $\dfrac{k}{s^2+k^2}$                              \\ [3.0ex]
            \Large $cos(kt)$                         &\Large      $\dfrac{s}{s^2+k^2}$                              \\ [3.0ex]
            \Large $senh(kt)$                        &\Large      $\dfrac{k}{s^2-k^2}$                              \\ [3.0ex]
            \Large $cosh(kt)$                        &\Large      $\dfrac{s}{s^2-k^2}$                              \\ [3.0ex]
            \Large $y'(t)$                           &\Large      $s[\mathscr{L}\{y(t)\}] -y(0)$                    \\ [3.0ex]
            \Large $y''(t)$                          &\Large      $s^2[\mathscr{L}\{y(t)\}] -sy(0) -y'(0)$          \\ [3.0ex]

            \Large $e^{at} \cdot t^n$                &\Large      $\dfrac{n!}{(s-a)^{n+1}}$                         \\ [3.0ex]
            \Large $e^{at} \cdot cos(kt)$            &\Large      $\dfrac{s-a}{(s-a)^2+k^2}$                        \\ [3.0ex]
            \Large $e^{at} \cdot sen(kt)$            &\Large      $\dfrac{k}{(s-a)^2+k^2}$                          \\ [3.0ex]


 
        \end{longtable}


    % =====================================================
    % ========    DEMOSTRACIONES DE LA TABLA       ========
    % =====================================================
    \clearpage
    \section{Demostraciones de la Tabla}

        % ==========================
        % ====       k        ======
        % ==========================
        \subsection{$f(t) = k$}
            Calcule la Tranformada de Laplace cuando $f(t) = k$

            Recuerda que podemos hacer que: $\mathscr{L}\{k\} = k \cdot \mathscr{L}\{1\}$
            \begin{equation*}
            \begin{split}
                \mathscr{L}\{1\} = F(s) 
                            & = \int_0^\infty e^{-st} 1 dt                                              \\
                            & = \lim_{b \to \infty} \int_0^b e^{-st} dt                                 \\
                            & = \lim_{b \to \infty} \frac{e^{-st}}{-s} |_0^b                            \\
                            & = \lim_{b \to \infty} \left[ \frac{1}{s} - \frac{e^{-sb}}{s} \right]      \\
                            & = \frac{1}{s}
            \end{split}
            \end{equation*}

            Por lo tanto:
            \begin{equation}   
                \mathscr{L}\{k\} = \frac{k}{s}
            \end{equation}

        % ==========================
        % ====    e^at        ======
        % ==========================
        \subsection{$f(t) = e^{at}$}
            Calcule la Tranformada de Laplace cuando $f(t) = e^{at}$

            \begin{equation*}   
            \begin{split}
                \mathscr{L}\{e^{at}\}
                    = F(s) & = \int_0^\infty e^{-st} \cdot e^{at} dt                                                 \\
                    & = \lim_{b \to \infty} \int_0^b e^{-st+at} dt                                                   \\
                    & = \lim_{b \to \infty} \int_0^b e^{-(s-a)t} dt                                                  \\
                    & = \lim_{b \to \infty} \frac{e^{-(s-a)t}}{-(s-a)} |_0^b                                         \\
                    & = \lim_{b \to \infty} \left[ \frac{e^{-(s-a)b}}{-(s-a)} -  \frac{e^{-(s-a)0}}{-(s-a)} \right]  \\
                    & = \frac{1}{s-a}
            \end{split}
            \end{equation*}


        % ==========================
        % ====    sen(kt)     ======
        % ==========================
        \subsection{$f(t) = sen(kt)$}
            Calcule la Tranformada de Laplace cuando $f(t) = sen(kt)$

            \begin{equation*}   
            \begin{split}
                \mathscr{L}\{sen(kt)\}
                    = F(s) & = \mathscr{L}\left\{ \frac{e^{kti} - e^{-kti}}{2i} \right\}             \\
                    & = \frac{1}{2i} \left(\mathscr{L}\{e^{kti}\}-\mathscr{L}\{e^{-kti}\}\right)     \\
                    & = \frac{1}{2i} \left( \frac{1}{s-ki} - \frac{1}{s+ki} \right)                  \\
                    & = \frac{1}{2i} \left( \frac{s+ki -(s-ki)}{s^2+k^2} \right)                     \\
                    & = \frac{1}{2i} \left( \frac{2ki}{s^2+k^2} \right)                              \\
                    & = \frac{k}{s^2+k^2} \\
            \end{split}
            \end{equation*}

        % ==========================
        % ====    cos(kt)     ======
        % ==========================
        \subsection{$f(t) = cos(kt)$}
            Calcule la Tranformada de Laplace cuando $f(t) = cos(kt)$

            \begin{equation*}   
            \begin{split}
                \mathscr{L}\{cos(kt)\}
                    = F(s) & = \mathscr{L}\left\{ \frac{e^{kti} + e^{-kti}}{2i} \right\}             \\
                    & = \frac{1}{2i} \left(\mathscr{L}\{e^{kti}\}+\mathscr{L}\{e^{-kti}\}\right)     \\
                    & = \frac{1}{2i} \left( \frac{1}{s-ki} + \frac{1}{s+ki} \right)                  \\
                    & = \frac{1}{2i} \left( \frac{s-ki+s+ki}{s^2+k^2} \right)                        \\
                    & = \frac{1}{2i} \left( \frac{2si}{s^2+k^2} \right)                              \\
                    & = \frac{s}{s^2+k^2}                                                            \\
            \end{split}
            \end{equation*}

        % ==========================
        % ====    senh(kt)    ======
        % ==========================
        \clearpage
        \subsection{$f(t) = senh(kt)$}
            Calcule la Tranformada de Laplace cuando $f(t) = senh(kt)$

            \begin{equation*}   
            \begin{split}
                \mathscr{L}\{senh(kt)\}
                    = F(s) & = \mathscr{L}\left\{ \frac{e^{kt} - e^{-kt}}{2i} \right\}             \\
                    & = \frac{1}{2} \left(\mathscr{L}\{e^{kt}\}-\mathscr{L}\{e^{-kt}\}\right)      \\
                    & = \frac{1}{2} \left( \frac{1}{s-k} - \frac{1}{s+k} \right)                   \\
                    & = \frac{1}{2} \left( \frac{s+k-(s-k)}{s^2-k^2} \right)                       \\
                    & = \frac{1}{2} \left( \frac{2k}{s^2-k^2} \right)                              \\
                    & = \frac{k}{s^2-k^2}                                                          \\
            \end{split}
            \end{equation*}


        % ==========================
        % ====    cosh(kt)    ======
        % ==========================
        \subsection{$f(t) = cosh(kt)$}
            Calcule la Tranformada de Laplace cuando $f(t) = cosh(kt)$

            \begin{equation*}   
            \begin{split}
                \mathscr{L}\{cosh(kt)\}
                    = F(s) & = \mathscr{L}\left\{ \frac{e^{kt} + e^{-kt}}{2i} \right\}             \\
                    & = \frac{1}{2} \left(\mathscr{L}\{e^{kt}\}-\mathscr{L}\{e^{-kt}\}\right)      \\
                    & = \frac{1}{2} \left( \frac{1}{s-k} + \frac{1}{s+k} \right)                   \\
                    & = \frac{1}{2} \left( \frac{s+k+s-k}{s^2-k^2} \right)                         \\
                    & = \frac{1}{2} \left( \frac{2s}{s^2-k^2} \right)                              \\
                    & = \frac{s}{s^2-k^2}                                                          \\
            \end{split}
            \end{equation*}


        % ==========================
        % ====    Derivadas   ======
        % ==========================
        \clearpage
        \subsection{$f(t) = g'(t)$}
            Calcule la Tranformada de Laplace cuando $f(t) = cosh(kt)$

            Recuerda que vamos a usar este cambio de variable:
            \begin{itemize}
                \item $u=e^{-st} \quad \to \quad du=-se^{-st}dt$
                \item $dv=g'(t)  \quad \to \quad v =g(t)$ 
            \end{itemize}

            \begin{equation*}   
            \begin{split}
                    \mathscr{L}\{g'(t)\} = F(s) 
                            & = \int_0^\infty e^{-st} g'(t) dt                                      \\
                            & = \to_{Aplicando Cambios}                                             \\
                            & = e^{-st} f(t)|_0^\infty - \int_0^\infty -se^{-st} g(t) dt            \\
                            & = e^{-st} f(t)|_0^\infty + s\int_0^\infty e^{-st} g(t) dt             \\
                            & = e^{-st} f(t)|_0^\infty + s\mathscr{L}\{g(t)\}                       \\
                            & = (e^{-s\infty} f(\infty))-(e^{-s0} f(0)) + s\mathscr{L}\{g(t)\}      \\
                            & = -f(0) + s\mathscr{L}\{g(t)\}                                        \\
                            & = s\mathscr{L}\{g(t)\} -f(0)                                          \\
            \end{split}
            \end{equation*}

    % =====================================================
    % ========          EJEMPLOS UTILES            ========
    % =====================================================
    \clearpage
    \section{Ejemplos Útiles}

        % ==========================
        % ====   EJEMPLOS     ======
        % ==========================
        \subsubsection{$f(t) = e^{t+7}$}
            Calcule la Tranformada de Laplace cuando $f(t) = e^{t+7}$

            \begin{equation*}   
            \begin{split}
                \mathscr{L}\{e^{t+7}\}  =                                                       \\
                            & = \mathscr{L} \{ e^t \cdot e^7 \}                                 \\
                            & = e^7 \cdot \mathscr{L} \{ e^t \}                                 \\
                            & = e^7 \frac{1}{s-1}                                               \\
                            & = \frac{e^7}{s-1}                                                 \\
            \end{split}
            \end{equation*}


% ======================================================================================
% ======================        ECUACIONES CON  LAPLACE       ==========================
% ======================================================================================
\chapter{Ecuaciones con Laplace}


    % =====================================================
    % ========          ECUACIONES CON LAPLACE     ========
    % =====================================================
    \clearpage
    \section{Ecuaciones de 2do Grado}

        % ==========================
        % ====   EJEMPLOS     ======
        % ==========================
        \subsubsection{$y'+ay=0$}
            Usando la Transformada de Laplace:

            \begin{equation*}   
            \begin{split}
                \mathscr{L}\{ y'-ay \} &= L\{0\}                            \\
                \mathscr{L}\{y'\}-a\mathscr{L}\{y\} &= L\{0\}               \\
                s\mathscr{L}\{y\}-y(0)-a\mathscr{L}\{y\} &= 0               \\
                (s-a)\mathscr{L}\{y\} &= y(0)                               \\
                \mathscr{L}\{y\} &= \frac{y(0)}{s-a}                        \\
                \mathscr{L}\{y\} &= y(0) \mathscr{L}\{e^{ax}\}              \\
                \mathscr{L}\{y\} &= \mathscr{L}\{y(0) \cdot e^{ax}\}        \\
                y &= y(0) \cdot e^{ax}                                      \\
            \end{split}
            \end{equation*}


        % ==========================
        % ====   EJEMPLOS     ======
        % ==========================
        \subsubsection{$y''+4y'+4y=t^2$}

            Donde $y(0) = y'(0) = 0$
            Usando la Transformada de Laplace:

            \begin{equation*}   
            \begin{split}
                \mathscr{L}\{ y''+4y'+4y \} &= L\{t^2\}                                                          \\
                \mathscr{L}\{y''\} +4\mathscr{L}\{y'\}+4\mathscr{L}\{y\} &= \frac{2!}{s^3} = \frac{2}{s^3}       \\
                s^2[\mathscr{L}\{y(t)\}] -sy(0) -y'(0)+4s[\mathscr{L}\{y(t)\}] -4y(0)+4\mathscr{L}\{y\} 
                                                                        &= \frac{2}{s^3}                         \\
                s^2[\mathscr{L}\{y(t)\}] +4s[\mathscr{L}\{y(t)\}]+4\mathscr{L}\{y\} &= \frac{2}{s^3}             \\
                \mathscr{L}\{y(t)\} (s^2+4s+4) &= \frac{2}{s^3}                                                  \\
                \mathscr{L}\{y(t)\}  &= \frac{\frac{2}{s^3}}{(s^2+4s+4)}                                         \\
                \mathscr{L}\{y(t)\}  &= \frac{2}{s^3 \cdot (s^2+4s+4)}                                           \\
                y(t)  &= \mathscr{L}^{-1} \left\{ \frac{2}{s^3 \cdot (s^2+4s+4)} \right\}                        \\
                y(t)  &= \mathscr{L}^{-1} \left\{ \frac{2}{s^3 \cdot (s^2+4s+4)} \right\}                        \\
                y(t)  &= \mathscr{L}^{-1} \left\{ \frac{2}{s^3(s+2)^2} \right\}                                  \\
            \end{split}
            \end{equation*}


            Ahora hagamos un paréntesis, tenemos ya solo que resolver esto por fracciones parciales:

            
% =====================================================
% ============        BIBLIOGRAPHY   ==================
% =====================================================
\clearpage
\bibliographystyle{plain}
	\begin{thebibliography}{9}

	% ============ REFERENCE #1 ========
	\bibitem{Sitio1} 
		ProbRob
		\\\texttt{Youtube.com}


	 

\end{thebibliography}



\end{document}
